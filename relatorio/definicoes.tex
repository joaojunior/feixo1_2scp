\section{Problema do Caminho Mais Curto Robusto Intervalar com Arrependimento Relativo}\label{sec:RSP-IRR}
Seja $G = (V,A)$ um grafo direcionado, no qual $V$ é um conjunto de vértices e $A$ é um conjunto de arcos, 
onde cada arco $(i,j) \in A$ tem o custo $c_{i,j}$ no intervalo de valores $[l_{ij},u_{ij}]$, 
com $l_{ij} \leq u_{ij}  \forall (i,j) \in A$. Defini-se um cenário $r \subset S$, onde $S$ é o conjunto infinito de todos os cenários, como 
uma fixação de valores $c_{ij}^r \in [l_{ij}, u_{ij}]$ para todos os arcos $(i,j)\in A$. Dado um nó de origem $s \in V$, um nó de destino
$t \in V$ e um caminho $P \subseteq A$ entre $s$ e $t$, define-se o desvio robusto relativo de $P$ no cenário $r \subset S$, denominado $d_P$, como 
a razão entre $(i)$ a diferença do custo do caminho $P$ no cenário $r$, representado por $c^r_P$ e o custo do menor caminho entre $s$ e $t$ no cenário $r$, representado
por $c^r_{P^*(r)}$, e $(ii)$ o custo do menor caminho entre $s$ e $t$ no cenário $r$. Como existem infinitos cenários, o caminho $P \subseteq A$ possui 
diversos desvio robusto relativo, um para cada cenário $r \subset S$. A partir da proposição 2.3 presente em \cite{karasan01}, podemos derivar a 
proposição 1, que mostra como é possível encontrar o desvio robusto relativo máximo de um caminho $P \subseteq A$.\\ 
\newtheorem{ambiente}{Proposição}
\begin{ambiente}
Dado um grafo $G = (V,A)$, o desvio robusto relativo para um caminho $P \subseteq A$ de $s$ até $t$, com $s,t \in V$ é máximo no cenário onde 
todos os arcos de $P$ estão com custos fixados no $u_{ij}$ e os demais arcos de $A$ estão com custos fixados no $l_{ij}$. \\
\textbf{Prova:} Dado $d^{*}_{P}$ o desvio robusto relativo máximo do caminho $P$ obtido no cenário $s_p$. Seja $s$ o cenário no qual as arestas $(i,j)$ 
de $P \subseteq A$ estão com os valores fixados no $u_{ij}$ e os demais arcos $(i,j)$ de $A$ estão com valores fixados em $l_{ij}$. Então: \\
\begin{center}
\begin{equation*}
d^{*}_{P} = \frac{c^{s_p}_{P} - c^{s_p}_{P^*(s_p)}}{c^{s_p}_{P^*(s_p)}} = 
\frac{\sum\limits_{(i,j) \in P \setminus P^*(s_p)} c^{s_p}_{ij} - \sum\limits_{(i,j) \in P^*(s_p) \setminus P} c^{s_p}_{ij}}{\sum\limits_{(i,j) \in P^*(s_p) \setminus P} c^{s_p}_{ij}}
\\
\leq \frac{\sum\limits_{(i,j) \in P \setminus P^*(s_p)} c^{s}_{ij} - \sum\limits_{(i,j) \in P^*(s_p) \setminus P} c^{s}_{ij}}{\sum\limits_{(i,j) \in P^*(s_p) \setminus P} c^{s}_{ij}} 
= \frac{c^{s}_{P} - c^{s}_{P^*(s_p)}}{c^{s}_{P^*(s_p)}} \leq \frac{c^{s}_{P} - c^{s}_{P^*(s)}}{c^{s}_{P^*(s)}}
\end{equation}
\end{center}
\end{ambiente}
Portanto $s$ é um cenário que também maximiza o desvio robusto relativo para o caminho $P$.
\\
O problema do Caminho Mais Curto Robusto Intervalar com Arrependimento Relativo(RSP-IRR) consiste em encontrar o caminho robusto $P \subseteq A$ do nó de origem
$s \in V$ ao nó de destino $t \in V$ que possui o menor desvio robusto relativo máximo. A seção \ref{sec:modelagem} apresenta a 
modelagem matemática para o RSP-IRR em grafos que não possuem ciclos, a seção \ref{sec:grafosCiclos} propõe uma modelagem 
matemática do RSP-IRR para grafos que possuem ciclos positivos e a seção \ref{sec:linearizacao} mostra uma linearização da modelagem matemática
apresentada na seção \ref{sec:modelagem}. 

\subsection{Modelagem Matemática para o RSP-IRR em grafos acíclicos}\label{sec:modelagem}
\section{Modelagem Matematica}\label{sec:modelagem} 
Essa seção apresenta a formulação para o $SCP$ proposta em \cite{Bertsimas05}. Para formular o SCP como um problema
de otimização inteira, é introduzida uma matriz de incidência $A$ de tamanho $mxn$ para a coleção de subconjuntos 
$M_j, \forall j \in N$, com as entradas dadas por: \
\begin{align}
    & a_{ij} = \left \{\begin{array}{ll} 1; & \textrm{se } i \in M_j \textrm{,} \nonumber \\
    0; & \textrm{caso contrário} \nonumber \label{eq:fluxo}
    \end{array}\right. \\
\end{align}
Nessa formulação, a função objetivo \eqref{eq:objetivo} minimiza o custo da cobertura procurada. 
Os valores constantes $c_j, \forall j \in N$ são os custos de cada subconjunto $M_j$. 
A variável de decisão $x_j$ é igual a $1$ quando $j \in F$ e $0$, caso contrário, onde $F \subset N$ é a cobertura procurada. \\
\begin{align}
    & \text{min } \sum_{j \in N} c_jx_j \label{eq:objetivo} \\
    & \text{Sujeito à:} \nonumber \\
    & Ax \ge 1 \\
    & x \in \{0,1\}^n \label{eq:binarias}
\end{align}

\subsection{Modelagem Matemática para o RSP-IRR em Grafos que possuem Ciclos Positivos}\label{sec:grafosCiclos}
No problema de caminho mais curto clássico a restrição \eqref{eq:fluxo} garante que as possíveis soluções para esse problema sejam um caminho do 
nó de origem $s$ ao nó de destino $t$. A função objetivo nesse caso, minimiza o custo desses 
possíveis caminhos, encontrando assim a solução para o problema e garantindo que essa solução não tenha ciclos, pois estamos considerando grafos sem
ciclos negativos. No RSP-IRR, a restrição de fluxo \eqref{eq:fluxo} continua garantindo que as possíveis soluções sejam um caminho do nó de origem
$s$ ao nó de destino $t$, porém a função objetivo \eqref{eq:objetivo} não garante que a solução não tenha ciclos positivos. A figura \ref{fig:grafo_1} apresenta um grafo com os custos dos arcos definidos em um intervalo de valores. Nesse grafo, considerando apenas caminhos
sem ciclos, os possíveis caminhos do nó de origem $s$ ao nó de destino $t$ são:$\{(s,1),(1,t)\}$ e $\{(s,2),(2,t)\}$, com o desvio robusto relativo máximo 
respectivamente igual á $\frac{20165}{120}=167.04$ e $\frac{20120}{165}=120.94$. Considerando caminhos com ciclos positivos, o caminho 
$\{(s,1),(1,s),(s,2),(2,t),(t,1),(1,t)\}$ com desvio robusto relativo máximo igual á $\frac{60388}{20120}=2.0014$ será a solução para o RSP-IRR. Portanto 
caso nenhuma restrição garanta a não existência de ciclos positivos no caminho robusto $P$ do nó de origem $s$ ao nó de destino $t$, um caminho 
com ciclo positivo poderá ser a solução para o RSP-IRR.
\begin{figure}[!h]
    \begin{center}
          \begin{tikzpicture}[>=latex',line join=bevel,]
            [scale=.8,auto=left]
            \node (1) at (100bp,100bp) [draw,circle] {$1$};
            \node (2) at (0bp,0bp) [draw,circle] {$2$};
            \node (s) at (0bp,100bp) [draw,circle,fill=black!10] {$s$};
            \node (t) at (100bp,0bp) [draw,circle,fill=black!10] {$t$};
            \draw [->] (s) -- node[fill=white,inner sep=1pt,sloped] {$[87,10087]$} (1);
            \draw [->] (1) edge [bend left] node[fill=white,inner sep=1pt,sloped] {$[16,10016]$} (s);
            \draw [->] (s) -- node [fill=white,inner sep=1pt,sloped] {$[84,10084]$} (2);
            \draw [->] (2) edge [bend left] node[fill=white,inner sep=1pt,sloped] {$[94,10094]$} (s);                        					
            \draw [->] (1) -- node [fill=white,inner sep=1pt,sloped] {$[78,10078]$} (t);
            \draw [->] (t) edge [bend left] node[fill=white,inner sep=1pt,sloped] {$[87,10087]$} (1);
   			\draw [->] (2) -- node [fill=white,inner sep=1pt,sloped] {$[36,10036]$} (t);
            \draw [->] (t) edge [bend left] node[fill=white,inner sep=1pt,sloped] {$[93,10093]$} (2);
          \end{tikzpicture}
    \end{center}
    \caption{Exemplo de um grafo com os custos dos arcos definido por um intervalo de valores.}
    \label{fig:grafo_1}
\end{figure}
Esta seção propõe uma maneira de modelar o RSP-IRR em grafos que possuem ciclos positivos.\\
Segundo Miller-Tucker-Zemlin \cite{MTZ1991} é preciso inserir variáveis $t_i \forall i \in V$ juntamente com as restrições topológicas \eqref{eq:eliminacao1}-\eqref{eq:eliminacao3} 
para a eliminação de ciclos de um caminho em um grafo que possui ciclos positivos. A restrição \eqref{eq:eliminacao1} garante que se o arco $(i,j) \in A$ 
estiver no caminho considerado, isso acontece quando $y_{i,j} = 1$, o valor da variável $t_i$ será menor do que o valor de $t_j$. Isso garante a 
não existência de um ciclo, pois seja $P = \{(s,1),(1,s),(s,2),(2,t)\}$ o caminho considerado. Ao considerar o arco $(s,1)$ a restrição \eqref{eq:eliminacao1}
pode ser reescrita como $t_s - t_1 \leq -1$ e ao considerar o arco $(1,s)$ pode ser reescrita como $t_1 - t_s \leq -1$. É fácil verificar que essas duas
inequações não podem ser satisfeitas simultaneamente, portanto o ciclo não pode ocorrer. \\
A formulação não linear para o RSP-IRR, cuja solução não contempla caminhos com ciclos 
positivos é formada pela função objetivo \eqref{eq:objetivo} e as restrições 
\eqref{eq:sp}-\eqref{eq:caminhos_positivos}, \eqref{eq:eliminacao1}-\eqref{eq:eliminacao3}.
\begin{align}
    & t_i - t_j + |V|y_{i,j} \leq |V| - 1, \forall (i,j) \in A \label{eq:eliminacao1} \\
    & t_s = 0 \label{eq:eliminacao2} \\
    & 0 \leq t_i \leq |V| \label{eq:eliminacao3}
\end{align}

\subsection{Linearização para o RSP-IRR}\label{sec:linearizacao}
Com exceção da função objetivo \eqref{eq:objetivo} mostrada na seção anterior, a modelagem matemática para o RSP-IRR é linear.
Baseando-se em \cite{Bisschop2005} está seção propõe uma linearização da função objetivo \eqref{eq:objetivo}, constituindo uma modelagem
matemática por programação linear inteira mista para o problema RSP-IRR. \\
Inicialmente, a função objetivo \eqref{eq:objetivo} é reescrita como: \\
\begin{align}
    & \text{Z = min } \frac{\sum\limits_{(i,j) \in A} u_{ij}y_{ij}}{x_t} - \frac{x_t}{x_t} \nonumber
\end{align}
Como $\frac{x_t}{x_t} = 1$, podemos reescrever a função objetivo anterior como:
\begin{align}
    & \text{Z = min } \sum\limits_{(i,j) \in A} u_{ij}\frac{y_{ij}}{x_t} - 1 \nonumber
\end{align}
Em seguida, são introduzidas as variáveis $z_{i,j}$, com a seguinte restrição:
\begin{align}
    & z_{ij} = \frac{y_{ij}}{x_t}, \forall (i,j) \in A \nonumber
\end{align}
Então, a função objetivo pode agora ser novamente reescrita como:
\begin{align}
    & \text{Z = min } \sum\limits_{(i,j) \in A} u_{ij}z_{ij} - 1 \label{eq:objetivolinear}
\end{align}
Para garantir que 
\begin{align}
    & z_{ij} = \frac{y_{ij}}{x_t}, \forall (i,j) \in A \nonumber
\end{align}
é preciso inserir as restrições \eqref{eq:linearizacao1},\eqref{eq:linearizacao2},\eqref{eq:linearizacao3}, abaixo.
\begin{align}
    & z_{ij} \leq y_{ij}, \forall (i,j) \in A  \label{eq:linearizacao1}\\
    & z_{ij} \geq \frac{1}{x_t} - (1 - y_{i,j}), \forall (i,j) \in A  \label{eq:linearizacao2}\\
    & z_{ij} \in [0,1], \forall (i,j) \in A \label{eq:linearizacao3}
\end{align}
Agora é preciso linearizar a restrição \eqref{eq:linearizacao2}, o que pode ser feito adicionando as variáveis $w_l$, tais que
$w_l = 1$, se e somente se $x_t = l$, onde $l \in [L,U]$ no qual $L$ é o custo do caminho mais curto entre o nó de origem $s$ e o nó de destino $t$ 
no cenário em que todos os arcos $(i,j) \in A$ estão com os valores fixados em $l_{i,j}$ e $U$ é o custo do caminho mais curto entre o nó de 
origem $s$ e o nó de destino $t$ no cenário em que todos os arcos $(i,j) \in A$ estão com os valores fixados em $u_{i,j}$. Para cada valor
$l \in [L,U]$, é introduzida uma variável $w_l \in \{0,1\}$. Essa variável tem o valor 1, quando $x_t = l$ e 0, caso contrário.
Portanto a restrição \eqref{eq:linearizacao2} pode ser reescrita como
\begin{align}
    & z_{ij} \geq \frac{1}{l}w_l - (1 - y_{ij}), \forall (i,j) \in A \label{eq:linearizacao4}
\end{align}
E as restrições \eqref{eq:linearizacao5},\eqref{eq:linearizacao6} e \eqref{eq:linearizacao7} são inseridas na formulação. 
\begin{align}
    & \sum\limits_{l \in [L,U]} w_l \geq 1  \label{eq:linearizacao5} \\
    & \sum\limits_{l \in [L,U]} lw_l = x_t  \label{eq:linearizacao6} \\
    & w_l \in \{0,1\} \forall l \in [L,U] \label{eq:linearizacao7}
\end{align}
Assim a formulação linear inteira mista para o RSP-IRR é constituída pela função objetivo \eqref{eq:objetivolinear} e as restrições 
\eqref{eq:sp}-\eqref{eq:caminhos_positivos}, \eqref{eq:linearizacao1},\eqref{eq:linearizacao3},\eqref{eq:linearizacao4}-\eqref{eq:linearizacao7}. Para
linearizar a modelagem matemática apresentada na seção \ref{sec:grafosCiclos} basta substituir a função objetivo pela função \eqref{eq:objetivolinear} 
e adicionar as restrições \eqref{eq:linearizacao1},\eqref{eq:linearizacao3},
\eqref{eq:linearizacao4}-\eqref{eq:linearizacao7} ao conjunto de restrições apresentadas naquela seção.


\section{Heurísticas para o RSP-IRR}\label{sec:heuristicas}
Segundo Averbakh \cite{Averbakh2005273}, o RSP-IRR é um problema NP-Difícil, portanto acredita-se que não pode ser resolvido de forma exata em tempo polinomial.
Essa seção apresenta duas heurísticas que serão comparadas. A seção \ref{sec:MKasperski} apresenta uma heurística desenvolvida 
por Kasperski e será chamada aqui de M-Kasperski. A seção \ref{sec:heuristicaProposta} apresenta uma 
heurística desenvolvida nesse trabalho e será chamada de Melhoria Percentual. \\

\subsection{Heurística M-Kasperski}\label{sec:MKasperski}
A heurística M-Kasperski [\cite{kasperski06},\cite{kasperski07},\cite{kasperski09}] busca o caminho mais curto robusto
entre $s \in A$ e $t \in A$ no cenário onde todas as arestas do grafo estão com o custo fixado na média aritmética do menor e maior
valor do intervalo dessas arestas. A heurística recebe como parâmetro de entrada um grafo $G=(V,A)$,
os limites inferiores $(l_{i,j})$, superiores $(u_{i,j})$ para o custo de cada arco $(i,j) \in A$ e o vértice de
origem $s \in V$ e de destino $t \in V$. A heurística M-Kasperski consiste em fixar o cenário de busca da solução
no cenário denominado $r^M$. Nesse cenário, cada arco $(i,j) \in A$ está com o custo fixado em $c^{r^M}_{i,j} = \frac{l_{ij}+u_{ij}}{2}$.
Após fixar o cenário $r^M$, a heurística M-Kasperski busca o caminho de menor custo entre o nó de origem $s$ e o nó de destino $t$ utilizando
o algoritmo de Dijkstra \cite{dijkstra59}. Portanto a ordem de complexidade dessa heurística no pior caso é $O(|A| + |V|log(V))$ \cite{dijkstra59}. \\
A figura \ref{f_Kasperski} apresenta o algoritmo para a heurística M-Kasperski.
A linha $2$ executa o algoritmo de
Dijsktra \cite{dijkstra59} para encontrar o menor caminho do nó de origem $s$ até o nó de destino $t$ no cenário $r^M$.
A linha $3$ retorna então o caminho robusto encontrado pela heurística M-Kasperski.
\begin{figure}
\centering
\includegraphics[width=4in]{AlgoritmoHeuristicaKasperski.PNG}
\caption{Algoritmo da Heurística M-Kasperski}
\label{f_Kasperski}
\end{figure}

\subsection{Heurística Melhoria Percentual}\label{sec:heuristicaProposta}
Essa heurística consiste na busca e melhoria do caminho mais curto entre o nó de origem $s \in A$ e o nó de destino $t \in A$ no cenário, $r^u$, 
onde todas as arestas $(i,j) \in A$ estão com os custos fixados em $u_{ij}$. Essa heurística recebe como parâmetro de entrada um grafo $G=(V,A)$,
os limites inferiores $(l_{i,j})$ e superiores $(u_{i,j})$ para o custo de cada arco $(i,j) \in A$, o vértice de
origem $s \in V$ e de destino $t \in V$ e calcula o caminho de menor custo, denominado $P^{r^u}$, entre $s$ e $t$ no cenário $r^u$. 
Após isso a heurística retira de $A$ uma porcentagem das arestas que estão no caminho $P^{r^u}$ e calcula o caminho de menor custo, encontrando 
um novo caminho $P^{r^u}$, entre $s$ e $t$ no cenário $r^u$. 
A heurística armazena o melhor desses dois caminhos encontrados. Isso é feito até que $90\%$ das arestas do melhor caminho $P^{r^u}$ tenha sido 
retirados de $A$ ou que cinco iterações aconteçam sem que o melhor caminho $P^{r^u}$ tenha sido alterado. \\
A figura \ref{f_HeuristicaProposta} apresenta o algoritmo dessa heurística. A linha 2 calcula o caminho de menor custo, $P^{r^u}$, entre 
o nó de origem $s \in A$ e o nó de destino $t \in A$ no cenário $r^u$. A linha 3 armazena esse caminho encontrado em $melhor\_caminho$. A 
linha 4 inicia a variável $iteracoes\_sem\_melhoria$ com o valor zero. O laço das linhas 5 até 18 é executado até 
que $90\%$ das arestas da melhor solução encontrada tenha sido retirada de $A$ ou que ocorrá
cinco iterações sem que a solução tenha sido melhorada. A linha 6 retira do grafo $G$ as primeiras $p\%$ arestas da melhor solução armazenada em 
$melhor\_caminho$. A linha 7 busca o caminho de menor custo entre o nó de origem $s$ e o nó de destino $t$ no cenário $r^u$ . A linha 8 
verifica se o custo do caminho encontrado é melhor que o custo da melhor solução armazenada, caso positivo, as linhas 9 e 10 são executadas, caso 
contrário a linha 13 é executada. A linha 9 substitui a melhor solução pela solução corrente e a linha 10 reinicia o valor de $iteracoes\_sem\_melhoria$ 
em zero. A linha 13 incrementa o valor de $iteracoes\_sem\_melhoria$.
A linha 19 retorna a melhor solução encontrada pela heurística. \\
Essa heurística executa o algoritmo de Dijkstra, no pior caso, dez vezes, fazendo a ordem de complexidade ser $O(|A| + |V|log(V))$.

\begin{figure}
\centering
\includegraphics[width=4in]{AlgoritmoHeuristicaProposta.PNG}
\caption{Algoritmo da Heurística Melhoria Percentual}
\label{f_HeuristicaProposta}
\end{figure}